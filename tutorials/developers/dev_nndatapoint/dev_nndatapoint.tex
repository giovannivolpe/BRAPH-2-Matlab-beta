\documentclass{tufte-handout}
\usepackage{../braph2_dev}
%\geometry{showframe} % display margins for debugging page layout

\title{Implement a new Neural Network Data Point}

\author[The BRAPH~2 Developers]{The BRAPH~2 Developers}

\begin{document}

\maketitle

\begin{abstract}
\noindent
This is the developer tutorial for implementing a new neural network data point. 
In this Tutorial, we will explain how to create the generator file \fn{*.gen.m} for a new neural network data point, which can then be compiled by \code{braph2genesis}. All kinds of neural network data point are (direct or indirect) extensions of the base element \code{NNDataPoint}. Here, we will use as examples the neural network data point \code{NNDataPoint\_CON\_REG} (connectivity data for regression), \code{NNDataPoint\_CON\_CLA} (connectivity data for classification) \code{NNDataPoint\_Graph\_REG} (adjacency matrix for regression), \code{NNDataPoint\_Graph\_CLA} (adjacency matrix for classification), \code{NNDataPoint\_Measure\_REG} (graph measure for regression), and \code{NNDataPoint\_Measure\_CLA} (graph measure for classification).
\end{abstract}

\tableofcontents

%%%%% %%%%% %%%%% %%%%% %%%%%
\clearpage
\section{Implementation of a Data Point with Connectivity Data}

\subsection{Connectivity Data Point for Regression (\code{NNDataPoint\_CON\_REG})}

We will start by implementing in detail \code{NNDataPoint\_CON\_REG}, which is a direct extension of \code{NNDataPoint}.
A data point for regression with connectivity data \code{NNDataPoint\_CON\_REG} contains the input and target for neural network analysis with a subject with connectivity data (SubjectCON), where the input is the subject's connectivity data and the target is the subject's variables of interest.

\begin{lstlisting}[
	label=cd:m:NNDataPoint_CON_REG:header,
	caption={
		{\bf NNDataPoint\_CON\_REG element header.}
		The \code{header} section of the generator code for \fn{\_NNDataPoint\_CON\_REG.gen.m} provides the general information about the \code{NNDataPoint\_CON\_REG} element.
		}
]
%% ¡header!
NNDataPoint_CON_REG < NNDataPoint (dp, connectivity regression data point) is a data point for regression with connectivity data.
 ¥\circled{1}\circlednote{1}{defines \code{NNDataPoint\_CON\_REG} as a subclass of \code{NNDataPoint}. The moniker will be \code{dp}.}¥

%%% ¡description!
A data point for regression with connectivity data (NNDataPoint_CON_REG) 
contains the input and target for neural network analysis with a subject with connectivity data (SubjectCON).
The input is the connectivity data of the subject.
The target is obtained from the variables of interest of the subject.
\end{lstlisting}


\begin{lstlisting}[
	label={cd:m:NNDataPoint_CON_REG:prop_update},
	caption={
		{\bf NNDataPoint\_CON\_REG element prop update.}
		The \code{props\_update} section of the generator code for \fn{\_NNDataPoint\_CON\_REG.gen.m} updates the properties of the \code{NNDataPoint\_CON\_REG} element. This defines the core properties of the data point.
	}
]
%% ¡props_update!

%%% ¡prop!
NAME (constant, string) is the name of a data point for regression with connectivity data.
%%%% ¡default!
'NNDataPoint_CON_REG'

%%% ¡prop!
DESCRIPTION (constant, string) is the description of a data point for regression with connectivity data.
%%%% ¡default!
'A data point for regression with connectivity data (NNDataPoint_CON_REG) contains the input and target for neural network analysis with a subject with connectivity data (SubjectCON). The input is the connectivity data of the subject. The target is obtained from the variables of interest of the subject.'

%%% ¡prop!
TEMPLATE (parameter, item) is the template of a data point for regression with connectivity data.
%%%% ¡settings!
'NNDataPoint_CON_REG'

%%% ¡prop!
ID (data, string) is a few-letter code for a data point for regression with connectivity data.
%%%% ¡default!
'NNDataPoint_CON_REG ID'

%%% ¡prop!
LABEL (metadata, string) is an extended label of a data point for regression with connectivity data.
%%%% ¡default!
'NNDataPoint_CON_REG label'

%%% ¡prop!
NOTES (metadata, string) are some specific notes about a data point for regression with connectivity data.
%%%% ¡default!
'NNDataPoint_CON_REG notes'

%%% ¡prop!  ¥\circled{1}\circlednote{1}{The property \code{INPUT} is the input value for this data point, which is obtained directly from the connectivity of \code{Subject\_CON} by the code under \code{¡calculate!}.}¥
INPUT (result, cell) is the input value for this data point.
%%%% ¡calculate!
value = {dp.get('SUB').get('CON')};
    
%%% ¡prop! ¥\circled{2}\circlednote{2}{The property \code{TARGET} is the target value for this data point, which is obtained directly from the variables of interest of \code{VOI\_DICT} by the code under \code{¡calculate!}.}¥
TARGET (result, cell) is the target value for this data point.
%%%% ¡calculate!
value = cellfun(@(x) dp.get('SUB').get('VOI_DICT').get('IT', x).get('V'), dp.get('TARGET_IDS'), 'UniformOutput', false);

\end{lstlisting}

\begin{lstlisting}[
	label={cd:m:NNDataPoint_CON_REG:props},
	caption={
		{\bf NNDataPoint\_CON\_REG element props.}
		The \code{props} section of generator code for \fn{\_NNDataPoint\_CON\_REG.gen.m} defines the properties to be used in \fn{NNDataPoint\_CON\_REG}.
	}
]
%% ¡props!

%%% ¡prop! ¥\circled{1}\circlednote{1}{The property \code{SUB} is a subject with connectivity data (\code{Subject\_CON}), which is used to calculated the mentioned properties \code{INPUT} and \code{TARGET}.}¥
SUB (data, item) is a subject with connectivity data.
%%%% ¡settings!
'SubjectCON'

%%% ¡prop! ¥\circled{2}\circlednote{2}{The property \code{TARGET\_IDS} defines the targets, where the targets should be from the subject's variable-of-interest IDs.}¥
TARGET_IDS (parameter, stringlist) is a list of variable-of-interest IDs to be used as regression targets.

\end{lstlisting}

\clearpage

\begin{lstlisting}[
	label=cd:m:NNDataPoint_CON_REG:tests,
	caption={
		{\bf NNDataPoint\_CON\_REG element tests.}
		The \code{tests} section from the element generator \fn{\_NNDataPoint\_CON\_REG.gen.m}.
		A test for creating example files should be prepared to test the properties of the data point. Furthermore, additional test should be prepared for validating the value of input and target for the data point.
	}
]			
%% ¡tests!

%%% ¡excluded_props!  ¥\circled{1}\circlednote{1}{List of properties that are excluded from testing.}¥
[NNDataPoint_CON_REG.SUB]

%%% ¡test!
%%%% ¡name!
Create example files for regression  ¥\circled{2}\circlednote{2}{creates the example connectivity data files for regression analysis.}¥
%%%% ¡code!
data_dir = [fileparts(which('NNDataPoint_CON_REG')) filesep 'Example data NN REG CON XLS'];
if ~isdir(data_dir)
    mkdir(data_dir);

    % Brain Atlas  ¥\circled{3}\circlednote{3}{creates and exports the brain atlas file to the example directory.}¥
    im_ba = ImporterBrainAtlasXLS('FILE', 'desikan_atlas.xlsx');
    ba = im_ba.get('BA');
    ex_ba = ExporterBrainAtlasXLS( ...
        'BA', ba, ...
        'FILE', [data_dir filesep() 'atlas.xlsx'] ...
        );
    ex_ba.get('SAVE')
    N = ba.get('BR_DICT').get('LENGTH');

    % saves RNG
    rng_settings_ = rng(); rng('default')

    sex_options = {'Female' 'Male'};

    % Group ¥\circled{4}\circlednote{4}{creates one group of subjects with specified degree and rewiring probability configurations.}¥
    K = 2; % degree (mean node degree is 2K)
    beta = 0.3; % Rewiring probability
    gr_name = 'CON_Group_XLS';
    gr_dir = [data_dir filesep() gr_name];
    mkdir(gr_dir);
    vois = [
        {{'Subject ID'} {'Age'} {'Sex'}}
        {{} {} cell2str(sex_options)}
        ];
    for i = 1:1:100 % subject number
        sub_id = ['SubjectCON_' num2str(i)];
        % create WS graphs with random beta
        beta(i) = rand(1); ¥\circled{5}\circlednote{5}{generates random rewiring probability settings for each subject.}¥
        h = WattsStrogatz(N, K, beta(i)); % create WS graph ¥\circled{6} \twocirclednotes{6}{10}{utilize the provided degree and rewiring probability settings to generate corresponding Watts-Strogatz model graphs.}¥

        A = full(adjacency(h)); A(1:length(A)+1:numel(A)) = 0; % extract the adjacency matrix
        r = 0 + (0.5 - 0) * rand(size(A)); diffA = A - r; A(A ~= 0) = diffA(A ~= 0); % make the adjacency matrix weighted
        A = max(A, transpose(A)); % make the adjacency matrix symmetric

        writetable(array2table(A), [gr_dir filesep() sub_id '.xlsx'], 'WriteVariableNames', false) ¥\circled{7}\circlednote{7}{exports the adjacency matrix of the graph to an Excel file.}¥

        % variables of interest
        age_upperBound = 80;
        age_lowerBound = 50;
        age = age_lowerBound + beta(i)*(age_upperBound - age_lowerBound); ¥\circled{8}\circlednote{8}{associates the age value with each individual rewiring probability setting.}¥
        vois = [vois; {sub_id, age, sex_options(randi(2))}];
    end
    writetable(table(vois), [data_dir filesep() gr_name '.vois.xlsx'], 'WriteVariableNames', false) ¥\circled{9}\circlednote{9}{exports the variables of interest to an Excel file.}¥

    % reset RNG
    rng(rng_settings_)
end
%%% ¡test_functions!
function h = WattsStrogatz(N, K, beta) ¥\circled{10}¥
% H = WattsStrogatz(N,K,beta) returns a Watts-Strogatz model graph with N
% nodes, N*K edges, mean node degree 2*K, and rewiring probability beta.
%
% beta = 0 is a ring lattice, and beta = 1 is a random graph.

% Connect each node to its K next and previous neighbors. This constructs
% indices for a ring lattice.
    s = repelem((1:N)', 1, K);
    t = s + repmat(1:K, N, 1);
    t = mod(t - 1, N) + 1;
    
    % Rewire the target node of each edge with probability beta
    for source = 1:N
        switchEdge = rand(K, 1) < beta;
        
        newTargets = rand(N, 1);
        newTargets(source) = 0;
        newTargets(s(t == source)) = 0;
        newTargets(t(source, ~switchEdge)) = 0;
        
        [~, ind] = sort(newTargets, 'descend');
        t(source, switchEdge) = ind(1:nnz(switchEdge));
    end
    
    h = graph(s,t);
end

%%% ¡test! 
%%%% ¡name! ¥\circled{11}\circlednote{11}{validates the data point by using assertions to confirm that the input and target calculated values match the connectivity data and the variables of interest in the example files.}¥
Create a NNDataset containg NNDataPoint_CON_REG with simulated data
%%%% ¡code!
% Load BrainAtlas
im_ba = ImporterBrainAtlasXLS( ...
    'FILE', [fileparts(which('NNDataPoint_CON_REG')) filesep 'Example data NN REG CON XLS' filesep 'atlas.xlsx'], ...
    'WAITBAR', true ...
    );

ba = im_ba.get('BA');

% Load Group of SubjectCON
im_gr = ImporterGroupSubjectCON_XLS( ...
    'DIRECTORY', [fileparts(which('NNDataPoint_CON_REG')) filesep 'Example data NN REG CON XLS' filesep 'CON_Group_XLS'], ...
    'BA', ba, ...
    'WAITBAR', true ...
    );

gr = im_gr.get('GR');

% create an item list of NNDataPoint_CON_REG ¥\circled{12}\threecirclednotes{12}{13}{14}{creates an item list for the data points, subsequently generates the data point dictionary using the list, and then constructs the neural network dataset containing these data points.}¥
it_list = cellfun(@(x) NNDataPoint_CON_REG( ...
    'ID', x.get('ID'), ...
    'SUB', x, ...
    'TARGET_IDS', x.get('VOI_DICT').get('KEYS')), ...
    gr.get('SUB_DICT').get('IT_LIST'), ...
    'UniformOutput', false);

% create a NNDataPoint_CON_REG DICT ¥\circled{13}¥
dp_list = IndexedDictionary(...
        'IT_CLASS', 'NNDataPoint_CON_REG', ...
        'IT_LIST', it_list ...
        );

% create a NNDataset containing the NNDataPoint_CON_REG DICT ¥\circled{14}¥
d = NNDataset( ...
    'DP_CLASS', 'NNDataPoint_CON_REG', ...
    'DP_DICT', dp_list ...
    );

% Check whether the number of inputs matches ¥\circled{14}\circlednote{14}{tests the number of inputs from the dataset matches the number of subjects in the group.}¥
assert(length(d.get('INPUTS')) == gr.get('SUB_DICT').get('LENGTH'), ...
		[BRAPH2.STR ':NNDataPoint_CON_REG:' BRAPH2.FAIL_TEST], ...
		'NNDataPoint_CON_REG does not construct the dataset correctly. The number of the inputs should be the same as the number of imported subjects.' ...
		)

% Check whether the number of targets matches ¥\circled{15}\circlednote{15}{tests the number of targets from the dataset matches the number of subjects in the group.}¥
assert(length(d.get('TARGETS')) == gr.get('SUB_DICT').get('LENGTH'), ...
		[BRAPH2.STR ':NNDataPoint_CON_REG:' BRAPH2.FAIL_TEST], ...
		'NNDataPoint_CON_REG does not construct the dataset correctly. The number of the targets should be the same as the number of imported subjects.' ...
		)

% Check whether the content of input for a single datapoint matches ¥\circled{16}\circlednote{16}{tests the value of each input from the data point matches the subject's connectivity data.}¥
for index = 1:1:gr.get('SUB_DICT').get('LENGTH')
    individual_input = d.get('DP_DICT').get('IT', index).get('INPUT');
    known_input = {gr.get('SUB_DICT').get('IT', index).get('CON')};

    assert(isequal(individual_input, known_input), ...
        [BRAPH2.STR ':NNDataPoint_CON_REG:' BRAPH2.FAIL_TEST], ...
        'NNDataPoint_CON_REG does not construct the dataset correctly. The input value is not derived correctly.' ...
        )
end

%%% ¡test! 
%%%% ¡name!  ¥\circled{17}\twocirclednotes{17}{18}{executes the corresponding example scripts to ensure the functionalities.}¥
Example training-test regression
%%%% ¡code!
% ensure the example data is generated
if ~isfile([fileparts(which('NNDataPoint_CON_REG')) filesep 'Example data NN REG CON XLS' filesep 'atlas.xlsx'])
    test_NNDataPoint_CON_REG % create example files
end

example_NN_CON_REG

%%% ¡test! 
%%%% ¡name! ¥\circled{18}¥
Example cross-validation regression
%%%% ¡code!
% ensure the example data is generated
if ~isfile([fileparts(which('NNDataPoint_CON_REG')) filesep 'Example data NN REG CON XLS' filesep 'atlas.xlsx'])
    test_NNDataPoint_CON_REG % create example files
end

example_NNCV_CON_REG

\end{lstlisting}

%%%%% %%%%% %%%%% %%%%% %%%%%
\clearpage
\subsection{Connectivity Data Point for Classification (\code{NNDataPoint\_CON\_CLA})}

%%%%% %%%%% %%%%% %%%%% %%%%%
\clearpage
\section{Implementation of a Data Point with Graphs}
\subsection{Graph Data Point for Regression (\code{NNDataPoint\_Graph\_REG})}

%%%%% %%%%% %%%%% %%%%% %%%%%
\clearpage
\section{Implementation of a Data Point with Graph Measures}
\subsection{Graph Measure Data Point for Classification (\code{NNDataPoint\_Measure\_CLA})}


%\bibliography{biblio}
%\bibliographystyle{plainnat}

\end{document}