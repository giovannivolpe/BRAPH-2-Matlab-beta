\documentclass{tufte-handout}
\usepackage{braph2_dev}
%\geometry{showframe} % display margins for debugging page layout

\title{Calculating Different Measures}

\begin{document}

\maketitle

\begin{abstract}
\noindent
This is the developer tutorial for adapting a script for calculating different graph measures. In this tutorial, we will explain how to edit an example script to calculate different graph measures in braph2genesis. Here, we use EXAMPLE\_ST\_MP\_WU as an example to show how to edit this script to calculate different graph measures for structural data using a multiplex weighted undirected graph.
\end{abstract}

\tableofcontents

%%%%% %%%%% %%%%% %%%%% %%%%%
\clearpage
\section{Atlas Loading}
We started with implementing the ImporterBrainAtlasXLS. The function allows users to import the atlas. 
\begin{lstlisting}[
	label=cd:m:degree:header,
	caption={
		{\bf Brain Atlas Importer.}
		The \code{header} section of the pipeline \fn{EXAMPLE\_ST\_MP\_WU}. It defines the importer for Atlas loading.
	}
]
%% ¡header!
% EXAMPLE_ST_MP_WU
% Script example pipeline ST MP WU

clear variables %#ok<*NASGU>

%% Load BrainAtlas
im_ba = ImporterBrainAtlasXLS( ... ¥\circled{1}\circlednote{1}{First, the script loads the brain atlas from the excel file atlas.xlsx}¥
    'FILE', [fileparts(which('example_ST_MP_WU')) filesep 'Example data ST_MP XLS' filesep 'atlas.xlsx'], ...
    'WAITBAR', true ...
    );

ba = im_ba.get('BA');
\end{lstlisting}





%%%%% %%%%% %%%%% %%%%% %%%%%
%\clearpage
\section{Data Loading}
The next step is to load the data of two groups.
\begin{lstlisting}[
	label=cd:m:degreeav:header,
	caption={
		{\bf Group Subject Data Importer.}
		The group data importer section provides the code for loading group subject data.
	}
]
im_gr1 = ImporterGroupSubjectST_MP_XLS( ...¥\circled{1}\circlednote{1}{ImporterGroupSubjectST\_MP\_XLS imports a group of subjects with structural data and their covariates (optional) from another XLS/XLSX file.}¥
    'DIRECTORY', [fileparts(which('SubjectST_MP')) filesep 'Example data ST_MP XLS' filesep 'ST_MP_Group_1_XLS'], ...
    'BA', ba, ...
    'WAITBAR', true ...
    );

gr1 = im_gr1.get('GR');

im_gr2 = ImporterGroupSubjectST_MP_XLS( ...¥\circled{1}¥
    'DIRECTORY', [fileparts(which('SubjectST_MP')) filesep 'Example data ST_MP XLS' filesep 'ST_MP_Group_2_XLS'], ...
    'BA', ba, ...
    'WAITBAR', true ...
    );

gr2 = im_gr2.get('GR');

\end{lstlisting}
\clearpage

\section{Group Analysis}
This section is to initialize the group data analysis.
\begin{lstlisting}[
	label=cd:m:degreeav:header,
	caption={
		{\bf Group Subject Data Analysis.}
		The group data analysis provides code for initialization of group data analysis.
	}
]
a_WU1 = AnalyzeGroup_ST_MP_WU( ...¥\circled{1}\circlednote{1}{The script initialize the group analysis with the loaded atlas and group data.}¥
    'GR', gr1, ...
    'CORRELATION_RULE', Correlation.PEARSON ...
    );

a_WU2 = AnalyzeGroup_ST_MP_WU( ...¥\circled{1}¥
    'TEMPLATE', a_WU1, ...
    'GR', gr2 ...
    );

\end{lstlisting}
%\clearpage
\section{Measure Calculation}
The last step is to calculate graph measures with data loaded in previous step. Here we use degree as an example.
\begin{lstlisting}[
	label=cd:m:degreeav:header,
	caption={
		{\bf Group Subject Data Analysis.}
		The group data analysis provides code for initialization of group data analysis.
	}
]

g_WU1 = a_WU1.memorize('G');¥\circled{1}\circlednote{1}{memorize in case there are measures with non-default rules.}¥
degree_WU1 = g_WU1.get('MEASURE', 'Degree').get('M');¥\circled{2}\circlednote{2}{The function calculates graph measure. In this case, Degree is calculated. \\
Other measures can also be calculated by changing the measure names}¥

g_WU2 = a_WU2.get('G');
degree_WU2 = g_WU2.get('MEASURE', 'Degree').get('M');¥\circled{2}¥
\end{lstlisting}



%%%%% %%%%% %%%%% %%%%% %%%%%
%\clearpage


\end{document}
